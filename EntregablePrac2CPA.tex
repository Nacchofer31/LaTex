\documentclass[11pt]{article}
\usepackage{listings}
\usepackage{color}
\usepackage{pgfplots}

\usepackage{geometry}
\geometry{legalpaper, margin=0.9in}


\definecolor{dkgreen}{rgb}{0,0.6,0}
\definecolor{gray}{rgb}{0.5,0.5,0.5}
\definecolor{mauve}{rgb}{0.58,0,0.82}

\lstset{frame=tb,
  language=Java,
  aboveskip=3mm,
  belowskip=3mm,
  showstringspaces=false,
  columns=flexible,
  basicstyle={\small\ttfamily},
  numbers=none,
  numberstyle=\tiny\color{gray},
  keywordstyle=\color{blue},
  commentstyle=\color{dkgreen},
  stringstyle=\color{mauve},
  breaklines=true,
  breakatwhitespace=true,
  tabsize=3
}

%Gummi|065|=)
\title{\textbf{Entregable Practica 2: Comparación de código genético}\\ }
\author{Ignacio Ferrer Sanz\\
		Alejandro Tauste Botella}
\begin{document}

\maketitle

\section*{Ejercicio 1}
Para la realización de este ejercicio se ha definido una sección paralela para determinar el número de hilos e imprimirlos por consola. Seguidamente se han obtenido los tiempos antes y después de la instrucción de \emph{buscar más parecidos}. De esta forma, obtenemos el tiempo que tarda en realizar la función. Todo esto importando las librerías necesarias de OpenMP.

\emph{El tiempo obtenido con una ejecución de 3 hilos es de 134 segundos}
\begin{lstlisting}
#include <omp.h>
#include <stdio.h>

					...
					
int hilos;
  #pragma omp parallel 
  {
    if(omp_get_thread_num() == 0) {
      	//Obtenemos el num. de hilos
      	hilos = omp_get_num_threads();
  		}
   }
					...

printf("N de hilos: %d\n", hilos);
t1 = omp_get_wtime();
busca_mas_parecidos(nt1,tabla1,nt2,tabla2,ps);
t2 = omp_get_wtime();

printf("Tiempo: %f\n", t2-t1);
escribe_resultado(salida,tabla1,nt2,ps,numeros);

\end{lstlisting}

\section*{Ejercicio 2}
Para el ejercicio 2 definimos una sections. Dentro de la sections definimos una section por cada fichero a cargar.

\emph{El tiempo obtenido con una ejecución de 3 hilos es de 134 segundos}
\begin{lstlisting}
#pragma omp parallel sections 
  {
	#pragma omp section
  	err  = lee_lineas(argv[1],&nt1,&tabla1);

  	#pragma omp section  
  	err2 = lee_lineas(argv[2],&nt2,&tabla2);
}
\end{lstlisting}

\pagebreak

\section*{Ejercicio 3}
Definimos una cláusula parallel for privatizando las variables d, dm, im e i; con el objetivo, de que estas variables tengan el valor correspondiente a cada hilo, evitando de esta forma que la ejecución del código presente condiciones de carrera.

\begin{lstlisting}

void busca_mas_parecidos(int nt1,char *tabla1[],int nt2,char *tabla2[],int ps[])
{ int i,j,d,im,dm;// dm es la distancia del codigo mas parecido, im su indice
                  // dm is the distance of the most similar code, im is its index
  
  #pragma omp parallel for private(d,dm,im,i)
    for ( j = 0 ; j < nt2 ; j++ ) {
      dm = MAX_LON + 1;
      for ( i = 0 ; i < nt1 ; i++ ) {
			d = distancia(tabla1[i],tabla2[j]);
			if ( d < dm ) {
			  dm = d;
			  im = i;
			}
       }
      ps[j] = im;
    }

\end{lstlisting}

Al realizar la orden por consola de: \emph{comp salida1.txt salida2.txt}, no se arrojan diferencias entre los dos archivos por tanto, la ejecución es correcta en ambos apartados.

\section*{Ejercicio 4}

Creamos una cláusula OpenMP, para paralelizar el bucle externo, privatizando las variables d e i. Adicionalmente, debemos crear una sección critica mediante la cláusula critical con el objetivo de que para cada hilo se realize la comprobación del if.

\emph{El tiempo obtenido con una ejecución de 4 hilos es de 34 segundos}

\begin{lstlisting}
void busca_mas_parecidos(int nt1,char *tabla1[],int nt2,char *tabla2[],int ps[])
{ int i,j,d,im,dm;// dm es la distancia del codigo mas parecido, im su indice
                  // dm is the distance of the most similar code, im is its index
  for ( j = 0 ; j < nt2 ; j++ ) {
    dm = MAX_LON + 1;
    #pragma omp parallel for private(d,i)
    for ( i = 0 ; i < nt1 ; i++ ) {
      d = distancia(tabla1[i],tabla2[j]);
      #pragma omp critical
      if ( d < dm ) {
        dm = d;
        im = i;
      }
    }
    ps[j] = im;
  }
}
\end{lstlisting}

\pagebreak

\section*{Ejercicio 5}

\end{document}
