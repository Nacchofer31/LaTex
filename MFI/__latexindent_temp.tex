\documentclass[12pt]{amsart}

    \addtolength{\hoffset}{-2.25cm}
    \addtolength{\textwidth}{4.5cm}
    \addtolength{\voffset}{-2.5cm}
    \addtolength{\textheight}{5cm}
    \setlength{\parskip}{0pt}
    \setlength{\parindent}{15pt}
    
    \usepackage{amsthm}
    \usepackage{amsmath}
    \usepackage{amssymb}
    \usepackage[utf8]{inputenc}
    \usepackage[colorlinks = true, linkcolor = black, citecolor = black, final]{hyperref}
    
    \usepackage{graphicx}
    \usepackage{multicol}
    \usepackage{ marvosym }
    \usepackage{wasysym}
    \usepackage{tikz}
    \usetikzlibrary{patterns}
    
    \newcommand{\ds}{\displaystyle}
    \DeclareMathOperator{\sech}{sech}
    
    
    \setlength{\parindent}{0in}
    \pagestyle{empty}
    \begin{document}
    \thispagestyle{empty}
    {\scshape Nacchofer31} \hfill {\scshape \large Apuntes MFI} \hfill {\scshape Tema \#1}
     
    \hrule
    \medskip
    
    \textbf{- Aplicación distribuida}: Colección de 
    componentes heterogéneos dispersos sobre una red de 
    computadores para realizar una función determinada.
    
    \begin{itemize}
    
    \item  Nodos heterogéneos autónomos que cooperan y presentan dependencias entre ellos, con requisitos para su ejecución y seguridad (privacidad, autenticación,...).
    \item  Objetivo: Proporcionar servicio a sus usuarios, para ello sus componentes necesitan ser instalados en varios ordenadores.
    
    \end{itemize}
    
    \textbf{- El despliegue} contiene todas las tareas de la gestión del ciclo de vida de una aplicación 
    informática que suceden tras su desarrollo, presentados en el proceso de despliegue de un servicio:
       
    
    \begin{itemize}
    
    \item  Instalación y activación de programas (Resolver 
    las dependencias del 
    software y entre los agentes, configurar software, establecer orden en que arrancan los componentes,...).
    \item  Desactivación, detener el sistema de forma ordenada.
    \item  Actualización, reemplazar componentes (ej. nueva versión). 
    \item  Adaptación (sin detener el servicio) tras un fallo o recuperación de un agente, cambios en los agentes, escalado.
    
    \end{itemize}
    
    \textbf{- Programas + Despliegue = Servicios} Como resultado, se obtienen componentes de la 
    aplicación, que se despliegan para llegar a ser un servicio.
    
    \smallskip
    
    \textbf{- Acuerdos de nivel de servicio (SLA)} pueden ser establecidos en el área de los servicios distribuidos entre el proveedor de servicios y sus clientes.
    
    \begin{itemize}
    
    \item  Funcionalidad facilitada por el proveedor se ajusta a las necesidades y requsitos de los clientes.
    
    \item  Rendimiento suele depender de la carga. Clientes solicitan nivel mínimo de rendimiento. Proveedor solicita a los clientes que no excedan cierto nivel máximo de carga.
    
    \item  Disponibilidad: Proveedor garantiza que el servicio estará disponible cuando los clientes lo necesiten. El nivel de disponibilidad (en porcentaje de tiempo que el servicio estará accesible para responder a  solicitud) deberá establecerse en el SLA.
    
    \end{itemize}
    
    \textbf{- Tipos de servicio} se pueden encontrar dos tipos:
    \begin{itemize}
    
    \item  Efímeros, servicios de sesiones breves e interactivas de uso en ordenadores personales, por un solo usuario. Ej: procesador de textos, navegadores web, intérprete de órdenes...
    
    \item  Persistentes, sercicios siempre disponibles, usados concurrentemente por multitud de usuarios con conexiones remotas. Ej: banca, administración y comercio electrónicos,...
    \end{itemize}
    
    \textbf{- Automatización del despliegue}, para despliegues de gran escala. \begin{itemize}
    
    \item  Cada componente de la aplicación distribuida tiene su propio BLOB de código, fichero que mantiene el programa a ejecutar por el componente y una plantilla de configuración. órdenes...
    
    \item  Configuración global de la aplicación. Mantiene un plan de interconexión entre componentes (lista de dependencias que deben resolverse, lista endpoints expuestos). Decide dónde colocar cada instancia. 
    \end{itemize}
    
    \textbf{- Configuración:} Uno de los pasos iniciales del despliegue es la generación del descriptor de despliegue.
    
    \begin{itemize}
    
    \item  Las plantillas de configuración de cada componente deben rellenarse tantas veces como instancias vaya a tener el componente.
    
    \item  La información para rellenar estas plantillas consiste en la resolución de dependencias.
    
    \item  Cada componente ejecuta cierto programa y ese programa exporta uno o más puntos de acceso y de dependencias que raramente podrán solucionarse en el proceso de compilación sino en el despliegue o la ejecución del programa (Resolución dinámica).
    
    \item No se requiere ningún cambio en la forma en que se escribe el código de las bibliotecas o el código de los programas que las utilicen (Transparencia para el programador).
    
    \end{itemize}
       
    \vfill
    
    \newpage
    
    {\bf - Los Contenedores} son herramientas que mantienen a otros componentes software, proporcionando un entorno aislado (protegido) y pueden establecer una correspondencia entre sus puntos de acceso y los del ordenador anfitrión.
    
    \begin{itemize}
    
    \item Gestionan las etapas del ciclo de vida de los componentes instalados en ellos, generando estos eventos del ciclo de vida:
    
    \subitem - Creación: órdenes para construir una imagen válida del componente.
    
    \subitem - Registro (initiate): registro de la imagen en el sistema de contenedores.
    
    \subitem - Inicio (start): Inicia la ejecución del componente en algún contenedor.
    
    \subitem - Parada: Ejecución del componente, después podrá ser reaunudado o finalizado.
    
    \subitem - Reconfiguración: Modifica el contenido de la imagen modificando sus módulos.
    
    \subitem - Destrucción: Elimina la imagen del componente del sistema de contenedores.
    
    \item  Aprovisionamiento (provisioning) consiste en la tarea de reservar la infraestructura necesaria para que una aplicación distribuida pueda funcionar.
    
    \subitem  Reservar recursos para cada instancia de componente (procesador + memoria + almacenamiento) y para la intercomunicación entre componentes.
    
    \subitem  La infraestructura suele concretarse en un pool de máquinas virtuales interconectadas, el componente y sus requisitos se implementan y ejecutan sobre una máquina virtual.
    
    \item  Inconveniente: Son menos flexibles que las máquinas virtuales, ya que el software de la instancia ha de ser compatible con el anfitrión y el aislamiento entre contenedores no es perfecta y esto puede provocar interferencias y problemas de seguridad.
    
    \item  Ventajas: Utiliza mucho menos recursos, consumen aproximadamente entre 10 y 100 veces menos recursos (reduce espacio). Si la parte inmutable se encuentra "precargada", se ahorra (90\%) para iniciar cada instancia (reduce tiempo). Mayor facilidad de despliegue (fichero de configuración). Aplicable en casi todos los escenarios.
    
    
    \end{itemize}
    
    {\bf - La inyección de dependencias} permite que el componente C sólo necesitará conocer la interfaz del servidor S para interactuar con él, que será implementada por una clase de proxy. El proxy P usado para interactuar con S es inyectado en C cuando se realiza el despliegue.
    \begin{itemize}
    
    \item  El proxy P usado para interactuar con S es inyectado en C cuando se realiza el despliegue.
    
    \item  Si el punto de acceso a S cambiara, el programa C no necesitará ser modificado, solo necesitaría una nueva versión del proxy P.
    
    \end{itemize}
    
    {\bf - Despliegue en la nube: IaaS}, en este modelo de despliegue:
    
    \begin{itemize}
    
    \item  La unidad a utilizar es la máquina virtual (MV). El proveedor ofrece un conjunto de tipos, según el tamaño y su capacidad de cómputo.
    
    \item  Presenta limitaciones, ya que este es demasiado primitivo:
    
    \subitem - No existen reglas que automaticen las decisiones de escalado de los componentes desplegados (bajo nivel).
    
    \subitem - No facilita herramientas para monitorizar el tráfico de red y seleccionar la ubicación de la MV.
    
    \subitem - Presenta un modelo de fallo insuficiente, los nodos de fallo no son realmente independientes y presenta una ayuda limitada frente a la recuperación.
    
    \end{itemize}
    
    {\bf - Despliegue en la nube: PaaS}, presenta mejoras de automatización del despliegue frente a las del modelo (IaaS):
    
    \begin{itemize}
    
    \item  Tiene como objetivo automatizar las tareas del ciclo de vida de los servicio.
    
    \item  La clave es el uso del SLA: El proveedor PaaS rellena un plan de despliegue y establece las reglas de escalabilidad a utilizar para obtener una adaptabilidad óptima, además de gestionar las actualizaciones del software de servicio respetando el SLA. 
    
    \item  Desafortunadamente los proveedores PaaS actuales no han alcanzado estos objetivos.
    La automatización es limitada, no se automatizan las decisiones de escalado con precisión, ya que no hay gestión del SLA ni actualizaciones.
    
    \end{itemize}
    
    {\bf - Docker}, ofrece una API para ejecutar procesos de forma aislada. 
    
    \begin{itemize}
    
    \item  Toma como base para construir una PaaS.
    
    \item  Soporta control de versiones (Git).
    
    \item  Define un sistema de ficheros de solo lectura para compartición entre contenedores.
    
    \item  Permite cooperación en el desarrollo mediante depósitos públicos.
    
    \item Presenta los siguientes componentes:
    
    \subitem - Imágenes (componente contructor): plantillas de solo lectura sobre las que se instancian contenedores.
    
    \$ docker build -t nombre\_imagen: Contruir una imagen a partir de un Dockerfile.
    
    \$ docker images: Muestra las imágenes creadas.
    
    \subitem - Depósito (componente distribuidor): depósito común para poder subir y compartir imágenes.
    
    \subitem - Contenedores (componente ejecutor): se crean a partir de imágenes, contienen todo lo necesario para ejecutarse.
    
    \$ docker run nombre\_imagen -i -t nombre\_contenedor: Tomar una imagen base y crear el contenedor.
    
    \$ docker commit id\_contenedor nueva\_imagen: Guardar ese estado como una nueva imagen.
    
    \end{itemize}
    
    {\bf - Órdenes en el archivo Dockerfile} 
    \begin{itemize}
    
    \item  FROM imageBase, primera instrucción (primera línea)
    
    \item MANITAINER nombre\_autor, establece el autor de la imagen.
    
    \item  ADD origen destino, Copia archivos de un lugar a otro. Origen suele ser un URL o un directorio (se copia el contenido) o un archivo accesible en el contexto de esa ejecución. Destino es una ruta en el contenedor.
    
    \item COPY origen destino, igual que ADD, pero no expande los ficheros comprimidos.
    
    \item RUN orden, ejecuta una orden (shell o exec), añadiendo un nuevo nivel sobre la imagen resultante.
    
    \item EXPOSE puerto, indica el puerto en el que el contenedor atenderá (listen) peticiones.
    
    \item  WORKDIR path, indica el directorio de trabajo para las órdenes RUN, CMD, ENTRYPOINT.
    
    \item USER uid, establece el UID bajo el que se ejecutará la imagen.
    
    \item ENV variable valor, asigna valores a las variables de entorno accesibles por los programas dentro del contenedor.
    
    \item VOLUME ruta\_contenedor, acceso del contenedor a un dir. del anfitrión, requiere run.
    
    \item CMD orden arg1 arg2 ..., proporciona los valores por defecto en la ejecución del contenedor.
    
    \item ENTRYPOINT orden arg1 arg2 ..., ejecuta dicha orden al crear el contenedor (terminal al finalizar la orden).
    
    
    IMPORTANTE: Sólo debería haber como máximo una orden CMD o ENTRYPOINT (si hay más, ejecuta la última).
    
    \end{itemize}
    
    {\bf - Múltiples componentes en distintos nodos} Docker admite enlaces entre contenedores de forma automatizada mediante docker-compose:
    
    \begin{itemize}
    
    \item Docker Compose, aplicación para definir y ejecutar aplicaciones ubicadas en varios contenedores Docker, limitado a contenedores de un único nodo, pero puede completarse con otro software de orquertación para controlar un cluster.
    
    Se requieren de tres pasos para el despliegue:
    
    \subitem 1. Definir el entorno de la aplicación con un Dockerfile.
    
    \subitem 2. Definir los  servicios de la aplicación en un archivo docker-compose.yml para que puedan ejecutarse conjuntamente.
    
    \subitem 3. Ejecutar  docker-compose  up,  con  lo que  Compose  iniciará  y ejecutará  la  aplicación completa.
    
    Las ordenes más significativas de docker compose son:
    
    \subitem - build: (re-)construye un servicio.
    
    \subitem - kill: detiene un contenedor.
    
    \subitem - start, stop, restart: inicia, detiene, reinicia un servicio.
    
    \subitem - rm: elimina un contenedor detenido.
    
    \subitem - run: ejecuta una orden en un servicio.
    
    \subitem - scale: número de contenedores a ejecutar para un servicio.
    
    \subitem - up: build + start.
    
    \subitem - port: muestra el puerto asociado al servicio.
    
    \subitem - ps: lista los contenedores.
    
    \subitem - pull: sube una imagen.
    
    Ciclo típico de uso:
    
    \subitem \$ docker-compose up -d
    
    \subitem \$ docker-compose stop
    
    \subitem \$ docker-compose rm -f
    
    En la creación de un descriptor de despliegue docker-compose.yml siguiendo la sintaxis YAML, los parámetros principales son:
    
    \subitem - image: referencia local o remota a una imagen, por nombre o tag.
    
    \subitem - build: ruta a un directorio que contiene un DockerFile.
    
    \subitem - command: cambia la orden a ejecutar en el inicio.
    
    \subitem - links: enlace a contenedores de otro servicio.
    
    \subitem - external\_links: enlace a contenedores externos a compose.
    
    \subitem - ports: puertos expuestos (en comillas "").
    
    \subitem - expose: Ídem, pero accesible sólo a servicios enlazados (mediante links).
    
    \subitem - volumes: monta rutas como volúmenes.
    
    \item Kubernetes, orquestador de contenedores ajeno a Docker, para distribuir las instancias entre los distintos nodos. Presenta los siguientes elementos principales:
    
    \subitem - Cluster y nodo (máquinas físicas y virtuales respectivamente).
    
    \subitem - Pod, unidad más pequeña desplegable que contiene un conjunto de contenedores con namespace y volúmenes compartidos.
    
    \subitem - Controladores de replicación, se encarga del ciclo de vida de un conjunto de pods, asegura que esté ejecutándose la cantidad especificada de réplicas del pod. Escalando, recuperando y replicando pods.
    
    \subitem - Controlador de despliegue, actualizan la aplicación distribuida.
    
    \subitem - Servicio, define un conjunto de pods y la forma de acceder a ellos.
    
    \subitem - Secretos, gestión de credenciales de nuestras aplicaciones.
    
    \subitem - Volúmenes, gestión de la presistencia de contenedores.
    
    \end{itemize}
    
    
    \pagebreak
    
    \end{document}