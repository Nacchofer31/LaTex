\documentclass[12pt]{amsart}

    \addtolength{\hoffset}{-2.25cm}
    \addtolength{\textwidth}{4.5cm}
    \addtolength{\voffset}{-2.5cm}
    \addtolength{\textheight}{5cm}
    \setlength{\parskip}{0pt}
    \setlength{\parindent}{15pt}
    
    \usepackage{amsthm}
    \usepackage{amsmath}
    \usepackage{amssymb}
    \usepackage[utf8]{inputenc}
    \usepackage[colorlinks = true, linkcolor = black, citecolor = black, final]{hyperref}
    
    \usepackage{graphicx}
    \usepackage{multicol}
    \usepackage{ marvosym }
    \usepackage{wasysym}
    \usepackage{tikz}
    \usetikzlibrary{patterns}
    
    \newcommand{\ds}{\displaystyle}
    \DeclareMathOperator{\sech}{sech}
    
    
    \setlength{\parindent}{0in}
    \pagestyle{empty}
    \begin{document}
    \thispagestyle{empty}
    {\scshape Nacchofer31} \hfill {\scshape \large Apuntes MFI} \hfill {\scshape Tema \#1}
     
    \hrule
    \medskip
    
    \textbf{- La lógica}, proporciona una formulación \textbf{simbólica} e \textbf{independiente del dominio} de las leyes del pensamiento humano.
    Este doble carácter hace posible mecanizar sus técnicas y métodos. 
    
    \begin{itemize}
    
    \item  PROBLEMA: 
    
    - Tiene un carácter \textbf{estático}, se desarrolló para estudiar objetos matemáticos  bien definidos y consistentes, se requieren
    formas más dinámicas (y menos perfectas) de lógica. 
    
    - Los métodos de la lógica resultan más \textbf{caros} en términos computacionales, es necesario reducir sus costes.
    \item  SOLUCIÓN: \textbf{Lógica computacional}, modela el conocimiento impreciso, incompleto, dinámico, distribuido. 
    Soporta el razonamiento aproximado, temporal, no monótono,...
    
    
    \end{itemize}
    
    \textbf{Lógicas para Aplicaciones software:}

    
    
    Restricciones: \emph{- Lógica ecuacional}
    
    Extensiones de la lógica: \emph{- Lógica difusa - Lógica Lineal - Lógica
    Modales}

    \textbf{- Lógica ecuacional:}
    
    \begin{itemize}
    
    \item  Subconjunto de la lógica de primer orden,
     se eliminan todas las conectivas lógicas. El símbolo de predicado
     es la (=).
    \item s=t, dos términos son semánticamente iguales, aunque son diferentes
     sintácticamente. \emph{Ej: ascii('0') = 48}
    \item Un conjunto E de ecuaciones con el mismo símbolo "raíz" f
    en las partes izquierdas de las ecuaciones se describe como
    "la definición" de f. \emph{even(0)=true - even(X)=even(X-2)}
    
    \end{itemize}
    
    \textbf{Lógica para la programación:}

    Restricciones - Estilo de Lenguaje: 
    
    - \emph{Lógica ecuacional - Funcional (Haskell)}
    
    - \emph{Lógica clasual - Relacional (Prolog)}
    
    Extensiones: 
    
    - \emph{Lógica many-sorted +tipos}

    - \emph{Lógica order-sorted +herencia}

    - \emph{Lógica (modal) temporal +concurrencia}

    - \emph{Lógica (modal) dinámica +objetos}
    
    \textbf{- Unificación de lógicas: Lógica de Reescritura (RWL)}, 
    lógica del cambio que permite especificar la dinámica de un
    sistema. 
    
    \begin{itemize}
    
    \item  Integración "sin costuras" de distintas características: funciones,
    y tipos, indeterminismo, concurrencia, reflexión y genericidad.
    
    \item  Marco unificado en el que pueden definir distintas lógicas: ecuacional, clasual, lineal.
    
    \item  Existe una lógica temporal, asociada a la RWL, estrictamente
    más potente que CTL/LTL: la temporal logic of rewriting (LTR).
    
    \end{itemize}
    
    \textbf{- Maude:}, implementa eficientemente la lógica
    RWL.  
     \begin{itemize}
    
        \item Soporta de forma natural la especificación formal / modelado
        / programación en un estilo funcional.
        
        \item  Distingue entre la parte concurrente y funcional.
        
        \item  Reescritura módulo listas, conjuntos, multiconjuntos,...
        mediante atributos ecuacionales.

        \item Genericidad y tipos ordenados de datos.

        \item Infraestructura para análisis y verficación formal
        (alcanzabilidad, model-checking, theorem proving).

        \item Reflexión como soporte al meta-modelado, ejecución
        simbólica y construcción rápida de herramientas de soporte.
        
        \end{itemize}
    \pagebreak

    \textbf{- Lenguaje Maude}

    \begin{itemize}
        \item Sintaxis: Basada en ecuaciones y reglas de reescritura
        (estilo Haskell, ML, Scheme o Lisp)
        \item Semántica: Basada en la Lógica de Reescritura (RWL), que
        modela funciones, concurrencia y objetos.
    \end{itemize}

    \textbf{- Fundamentos de Maude}, consta de tres tipos de módulos:

    \begin{itemize}
        \item Módulos funcionales \emph{fmod \textless conjunto de ecuaciones \textgreater endfm}
        \item Módulos de sistema \emph{mod \textless conjunto de ecuaciones/reglas de escritura \textgreater endm}
        \item Módulos O2 \emph{omod ... endom}
    \end{itemize}

    \textbf{- E (Ecuaciones):}
    \begin{itemize}
        \item Definen funciones confluentes y terminantes.
        \item Se definen dentro de módulos funcionales.
        \item E puede incluir un conjunto Ax de Axiomas algebraicos.
        \item Representan la parte estática del sistema.
        \item Se aplican de forma determinista.
    \end{itemize}

    \textbf{- R (Reglas de reescritura):}
    \begin{itemize}
        \item Definen funciones que pueden ser no confluentes y/o no
        terminantes.
        \item Se definen dentro de módulos de sistema.
        \item Especifican la dinámica del sistema, es decir, acciones que
        puedan producir transiciones del mismo.
    \end{itemize}

    \textbf{- Paso de reescritura (Maude step)}, dado un término (o estado) s,
    un paso de reescritura de t a t' se consigue aplicando una regla de R
    módulo las ecuaciones de E. 

    El estado t se simplifica usando las ecuaciones de E hasta
    alcanzar su forma irreducible (tE) con respecto E.

    Una traza de ejecución es una secuencia de Maude steps.
    
    \end{document}